\documentclass[11pt,onecolumn,superscriptaddress,notitlepage]{article}

\usepackage[total={6.5in,9in}, top=1.0in, includefoot]{geometry}
\usepackage{epsfig}
\usepackage{subfigure}
\usepackage{placeins}
\usepackage{amsmath}
\usepackage[usenames,dvipsnames,svgnames,table]{xcolor}
\usepackage{amssymb}
\usepackage{setspace}
\usepackage{graphicx} % Include figure files
\usepackage{times}
\usepackage{amsthm}
\usepackage{hyperref}
\usepackage[affil-it]{authblk} 
\hypersetup{bookmarks=true, unicode=false, pdftoolbar=true, pdfmenubar=true, pdffitwindow=false, pdfstartview={FitH}, pdfcreator={Daniel Larremore}, pdfproducer={Daniel Larremore}, pdfkeywords={} {} {}, pdfnewwindow=true, colorlinks=true, linkcolor=red, citecolor=Green, filecolor=magenta, urlcolor=cyan,}

\usepackage{enumitem}

\newcommand{\dx}[0]{\displaystyle\frac{d}{dx}}
\newcommand{\dy}[0]{\displaystyle\frac{dy}{dt}}
\newcommand{\so}[1]{\textcolor{red}{#1}}

\usepackage{parskip}

\date{}
\begin{document}

%%%%%%%%%% Authors
\author{CSCI 2897 - Calculating Biological Quantities - Larremore - Fall 2021}
%%%%%%%%%% Title
\title{Homework 2}
%%%%%%%%%% Abstract
\maketitle
%%%%%%%%%% Content

    %%%    
    %%%   
  %%%%%%%
   %%%%%
    %%%
     %
{\bf Notes:} Remember to (1) familiarize yourself with the collaboration policies posted on the Syllabus, and (2) turn in your homework to Canvas as a {\bf single PDF}. Hand-writing some or most of your solutions is fine, but be sure to scan and PDF everything into a single document. Unsure how? Ask on Slack! 

\section*{Pushups}

{\bf Calculate these indefinite integrals. Don't forget your constant!} 

\begin{enumerate}
	\item $\displaystyle\int x^2\ dx=$
	\item $\displaystyle\int x^{-2}\ dx = $
	\item $\displaystyle\int e^{2 \pi x}\ dx = $
	\item $\displaystyle\int \sin{x}\ dx = $
	\item $\displaystyle\int \sin{x}\ \csc{x}\ dx = $
\end{enumerate}

\section*{Squats} 

{\bf For each family of solutions below, use the initial condition to solve for the unknown constant $\alpha$, and then write the solution with the solved-for constant plugged in and simplified.}

\begin{enumerate}[resume]
	\item $y(t) = \alpha e^{3t}, \quad y(0) = 10 $
	\item $y(t) = \alpha e^{t/2}, \quad y(6) = e $
	\item $\displaystyle n(t) = \frac{K}{1+\alpha Ke^{-r t}}, \quad n(0) = 1$
	\item $\displaystyle n(t) = \frac{K}{1+ \alpha Ke^{-r t}}, \quad n(0) = K$
	\item $\displaystyle y(x) = \frac{1}{x^2 + \alpha}, \quad y(2) = \frac{1}{3}$  
\end{enumerate}

\section*{Stay Hydrated} 
Torricelli's Law explains how fast water flows out of a tank. Imagine a cylindrical tank, oriented with the flat parts of the cylinder on bottom and top, in which the water is filled up to height $h$ meters (m), and suppose there is a hole in the bottom of the tank. Suppose that the tank has cross sectional area $A_T$ and the hole has cross sectional area $A_H$, both in units $m^2$. The top of the tank is open. Torricelli's law tells us that the {\it speed of the water} leaving the tank is $v = \sqrt{2 g h}$ m/s, where $g$ is a constant and $h$ is, again, the height of the water in the tank.  

But there is also an open faucet above the tank, and water is flowing into the top of the tank at a constant rate of $\frac{1}{10}$ m$^3$/s. In other words, water is flowing in at a constant rate, and water is flowing out at a rate that depends on Torricelli's law. 

Our goal is to be able to describe (1) the volume of water $V(t)$ in the tank, and (2) the height of the water $h(t)$ in the tank. 

\begin{enumerate}[resume]
	\item If water flows out according to Torricelli's law, and the area of the hole is $A_H$, what is the {\it volume} of water flowing out of the tank per second?\footnote{Hint: volumetric flow has units of m$^3$/s. Area has units m$^2$. Speed has units m/s.}
	\item If the tank is cylindrical, the height of the water is $h$, and the tank's cross sectional area is $A_T$, what is the volume of water in the tank as a function of the height of water $h$?
	\item Draw a picture of the tank, the faucet above, and the flow out the bottom. Label the flows in and out, the height of the water in the tank, and anything else you know.  {\it Next to} your drawing, draw a {\it flow diagram} for the volume of water in the tank $V$.  
	\item Write a differential equation for the rate of change of the volume of water in the tank with respect to time, $dV/dt$.
	\item Use this to write a differential equation for the rate of change of the {\it height} of water in the tank with respect to time, $dh/dt$. 
	\item Find an equilibrium solution for the height of water in the tank and explain your results in words.
\end{enumerate}

\clearpage
\section*{Separation of Variables}

{\bf Classify each equation as separable or not separable.}  Then, for separable equations, separate the variables {\it but do not integrate.}
\begin{enumerate}[resume]
	\item $t \dy = 4y$
	\item $t \dy = 4+t$
	\item $y \ln t \dy = \left(\frac{y+1}{t}\right)^2$
	\item $e^t + e^y = \dy$
	\item $e^{t+y} = \dy$
	\item Your equation from Question 15.
\end{enumerate}

{\bf Solve these initial value problems by using Separation of Variables.}\footnote{Hint: (1) separate, (2) integrate, (3) don't forget that constant of integration, (4) plug in the initial values to solve for the constant, (5) put it all together.}
\begin{enumerate}[resume]
	\item $\dy = \frac{y-1}{t-1}, \quad y(5)=2$
	\item $t^2 \dy = y-ty,\quad y(-1) = -1$
	\item $\dy = \frac{2t + 1}{2y},\quad y(-2) = 1$
\end{enumerate}

{\bf Extra Credit.} Find a solution of $\displaystyle t \dy = y^2 - y$ that passes through $(t,y) = (\tfrac{1}{2},\tfrac{1}{2})$

     %
    %%%
   %%%%%
  %%%%%%%
    %%%
    %%%

\end{document}