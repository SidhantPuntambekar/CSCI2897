\documentclass[11pt,onecolumn,superscriptaddress,notitlepage]{article}

\usepackage[total={6.5in,9in}, top=1.0in, includefoot]{geometry}
\usepackage{epsfig}
\usepackage{subfigure}
\usepackage{placeins}
\usepackage{amsmath}
\usepackage[usenames,dvipsnames,svgnames,table]{xcolor}
\usepackage{amssymb}
\usepackage{setspace}
\usepackage{graphicx} % Include figure files
\usepackage{times}
\usepackage{amsthm}
\usepackage{hyperref}
\usepackage[affil-it]{authblk} 
\hypersetup{bookmarks=true, unicode=false, pdftoolbar=true, pdfmenubar=true, pdffitwindow=false, pdfstartview={FitH}, pdfcreator={Daniel Larremore}, pdfproducer={Daniel Larremore}, pdfkeywords={} {} {}, pdfnewwindow=true, colorlinks=true, linkcolor=red, citecolor=Green, filecolor=magenta, urlcolor=cyan,}

\usepackage{enumitem}

\newcommand{\dx}[0]{\displaystyle\frac{d}{dx}}
\newcommand{\dy}[0]{\displaystyle\frac{dy}{dt}}
\newcommand{\so}[1]{\textcolor{red}{#1}}

\usepackage{parskip}

\date{}
\begin{document}

%%%%%%%%%% Authors
\author{CSCI 2897 - Calculating Biological Quantities - Larremore - Fall 2021}
%%%%%%%%%% Title
\title{Homework 4}
%%%%%%%%%% Abstract
\maketitle
%%%%%%%%%% Content

    %%%    
    %%%   
  %%%%%%%
   %%%%%
    %%%
     %
{\bf Notes:} Remember to (1) familiarize yourself with the collaboration policies posted on the Syllabus, and (2) turn in your homework to Canvas as a {\bf single PDF}. Hand-writing some or most of your solutions is fine, but be sure to scan and PDF everything into a single document. Unsure how? Ask on Slack! 

\section*{Burpees}

{\bf Write the integrating factor $\mu(t)$ for each of these 1st order linear ODEs. Recall that the first pages of Zill feature a nice table of integrals if you see something you're unsure how to integrate!} 

\begin{enumerate}
	\item $\dy + y = t+3$
	\item $\dy + 2y = t+3$
	\item $\dy + 2ty = t+3$
	\item $\dy + 2ty = (t+3)^2$
	\item $\dy = q(t) + \ln(t)y$
\end{enumerate}

\section*{Side planks} 

{\bf For each 1st order linear ODE below, use the integrating factor method to arrive at a solution for $y(t)$.}

\begin{enumerate}[resume]
	\item $\dy = t + \frac{y}{t}, \quad y(2) = 5$
	\item $\dy - e^{-2t} = 5y, \quad y(0) = \pi$
\end{enumerate}

\clearpage
\section*{Vaccination, Birth, and Death} 

This problem will focus a variation on the classic SIR model. We'll learn about {\bf the impact of birth and death} on vaccine-induced herd immunity by thinking about population turnover. 

Consider our typical $SIR+V$ model with a {\bf perfectly protective vaccine}, with the inclusion of birth and death. Specifically, suppose that a fraction $\omega$ of the total population dies per day. Suppose also that an equal number of people are born each day, so that the total population size is a constant. Assume that all people are {\it born susceptible $S$,} but that people in the $S$, $I$, $R$, and $V$ groups die at equal per-capita rates.

Finally, assume that at a constant rate $\alpha$, susceptibles are vaccinated. Note carefully that $\alpha$ is {\it not} a per-capita rate. 

\begin{enumerate}[resume]
	\item Using the typical parameters $\beta$ and $\gamma$ as introduced in class, draw a {\bf flow diagram} for this system. Use one color to draw the typical $SIR+V$ model part of the flow diagram, and use a {\bf second color} to show, in the same diagram, the birth and death modifications that we have introduced.
	\item Use your flow diagram to write the set of differential equations for this system.
\end{enumerate}	

The next few problems focus on understanding the equilibria in this system.
\begin{enumerate}[resume]	
	\item Begin your quest to find equilibria with your equation for $\dot{V}$. What is the steady-state value of $V$?
	\item Under what mathematical conditions does your equilibrium make sense? After stating this requirement in math, explain what it means in words in a single sentence. 
	\item Now turn to your equation for $\dot{R}$. What is the steady state equation you get from setting $\dot{R}=0$?
	\item Now focus on your equation for $\dot{I}$. Use this equation to find a disease-free equilibrium for the system, and specify the values of $S$, $I$, $R$, and $V$ at that equilibrium. 
	\item Explain the values of the disease-free equilibrium in words. 
	\item Using your remaining equations, what is the other possible equilibrium? In other words, what are the other possible steady-state values of $S$, $I$, $R$, and $V$ that are not ``disease free'' ?
	\item Under what mathematical conditions does this equilibrium make sense? What would happen if you tried to simulate from a model in which the parameters violated your mathematical conditions? 
\end{enumerate}

We'll now turn to exploring this model via simulation. 
\begin{enumerate}[resume]
	\item Modify the SIR code from the in-class notebook to include the effects of vaccination, birth, and death. Make a figure using the following constraints:
	\begin{itemize}
		\item Initially, let $1/1000$ of the population be infected, with everyone else susceptible. 
		\item Let $\beta=0.5$, $\alpha=0.001$, $\omega=0.002$, and $\gamma = 0.25$.
		\item Plot five years of simulation with $S$ in blue, $I$ in red, $R$ in black, and $V$ in green. 
	\end{itemize}
	\item Describe the epidemic in this plot in words, and discuss which of the two equilibria (which you found in previous questions) you think this system is heading toward.
	\item Now simulate anew and make a figure using the following constraints:
	\begin{itemize}
		\item Initially, let $1/1000$ of the population be infected, with everyone else susceptible. 
		\item Let $\beta=1$, $\alpha=0.001$, $\omega=0.002$, and $\gamma = 0.25$.
		\item Plot five years of simulation with $S$ in blue, $I$ in red, $R$ in black, and $V$ in green. 
	\end{itemize}
	\item Describe the epidemic in this plot in words. What is going on here? Again, discuss which of the two equilibria you think this system is heading toward. 
	\vspace{1in}
	\item [Extra Credit A] Using math or your code, explore the conditions under which the system goes to the disease-free equilibrium or the alternative. Try to write down general rules that could help someone understand what happens when a disease is spreading in a population with birth, death, and vaccination. 
	\item [Extra Credit B] Critique this model by commenting on its assumptions. How might you address some of those critiques with a modified model? 
\end{enumerate}

     %
    %%%
   %%%%%
  %%%%%%%
    %%%
    %%%

\end{document}